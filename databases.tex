\chapter{FileCatalog connected to a NoSQL database}
One of the main goals of this project was to test whether connecting the 
file catalog, more specifically its metadata part, to a NoSQL database would 
improve the feedback speed of the service thus making it more pleasant to use
or make it easier to implement and maintain. The database had to satisfy the
following conditions in order to be connectable to DIRAC and deployable in
the computing centers.

\begin{itemize}
\item There has to be a free version which will DIRAC be using.
\item The database has to have a python interface or client.
\end{itemize}

The characteristics of the data itself add some restrictions. There are two
ways the data would be extracted from the database. For directories 
the database has to be able to retrieve all the metadata associated with a directory
with a certain ID. The files are queried based on the terms in the metaquery so
all the querying techniques the metaquery could use have to be possible, including
range queries. 

\section{Apache Cassandra}


\section{MongoDB}

\section{Elasticsearch}

