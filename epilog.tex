\chapter*{Conclusion}
\addcontentsline{toc}{chapter}{Conclusion}
\label{chap:conc}

This project successfully added dataset support to the DIRAC File Catalog making DIRAC even more 
versatile. Dataset support was actively requested by two experiments, that can now start using DIRAC. 
Closely coupled with the dataset functionality went the development of the new MetaQuery
class that extended the metaquery language by adding more logical operators. The new MetaQuery also
supports normalization and optimization of the query input opening new possibilities for future usage
in the DIRAC Data Management system as well as the Workflow Management system. 

This project also tackled the problem of storing metadata. Trying to enhance the current solution a couple of NoSQL 
databases were tested on sample data similar to the DIRAC production metadata. The tests proved that connecting the 
metadata part of DFC to a NoSQL database could improve query performance, especially on more complex
queries. A better performance is observable when applying 4 and more constraints and retrieving less than 10 000 
hits.

To prove the concept of connecting a NoSQL database to DIRAC a new module was developed to provide 
a interface between DFC and the database. As the back-end database Elasticsearch was used, because it performed the 
best in the conducted tests. To improve the query performance, the complexity was moved from the python code of
DIRAC to the database engine. This was traded by adding complexity to the management procedures. These do not have 
to be, unlike the query mechanism, optimized for time performance.

In future if the DIRAC collaboration decides to use Elasticsearch as the database back-end for the metadata
catalog, more functionality can be added using Elasticsearch specific features. 
For example when a query is executed, the number of documents satisfying it is returned before all the documents are 
fetched. This could be used for example when checking the properties of a dynamic dataset or when trying to predict
the time for fetching all the results of a particular query could be.Also new comparison 
operators translating to one of Elasticsearch query 
types\footnote{\url{https://www.elastic.co/guide/en/elasticsearch/reference/current/term-level-queries.html}} can be 
added to extend the metaquery language.