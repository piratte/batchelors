\chapter*{Conclusion}
\addcontentsline{toc}{chapter}{Conclusion}

This project successfully added dataset support to the DIRAC File Catalog making DIRAC even more 
versatile. Closely coupled with the dataset functionality went the development of the new MetaQuery
class that extended the metaquery language by adding more logical operators. The new MetaQuery also
supports normalization and optimization of the query input opening new possibilities for future usage
in the DIRAC Data Management system as well as the Workflow Management system. 

This project also tackled the problem of storing metadata. To solve it a couple of NoSQL databases were tested
on sample data similar to the DIRAC production metadata. The tests proved that connecting the 
metadata part of DFC to a NoSQL database could improve query performance, especially on more complex
queries. 

To prove the concept of connecting a NoSQL database to DIRAC a new module was developed to provide 
a interface between DFC and the database. As the back-end database Elasticsearch, the database performing 
best in the conducted tests was used. This conveniently moved the complexity from query execution to the database 
and metadata management.