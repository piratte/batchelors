\chapter{The Distributed Infrastructure with Remote Agent Control}

\section{About DIRAC}
The LHCb Collaboration\cite{LHCb} is running one of the four detectors attached to the LHC particle 
collider at CERN, Geneva. The amount of data produced by the experiment
annually is so large that it necessitates the development of a specialized system for the data
production, reconstruction and analysis. The DIRAC project of the LHCb Collaboration was started to
provide such a system.\cite{Dir2} The developers were aiming to create a easy to run system, seamlessly using 
the various heterogeneous computing resources available to the LHCb Collaboration, that can be run by only one 
production manager. 

The DIRAC software architecture is based on a set of distributed, collaborating services. Designed to have a
light implementation, DIRAC is easy to deploy, configure and maintain on a variety of platforms. Following
the paradigm of a Services Oriented Architecture (SOA), DIRAC is lightweight, robust and scalable. Because one 
of the primary goals was to support various virtual organization with their specific needs, it supports well 
isolated plugable modules, where the organizations specific features can be located. It allows to construct medium
sized grids of up to several thousands processors by integrating diverse resources within its integrated Workload
Management System. The DIRAC Data Management components provide access to standard grid storage systems based 
on the SRM standard interface \cite{SRM} or ordinary (S)FTP, HTTP file servers. The File Catalog options include the
LCG File Catalog (LFC) as well as a simple DIRAC File Catalog, discussed later. 

\subsection{DIRAC architecture}
DIRAC components can be grouped in to 4 categories: + [figure] %TODO
\begin{description}

\item[Resources] \hfill \\
Resources are components which provide access to the computing and storage facilities available to
DIRAC. Computing resources include individual PCs, computer farms, cloud resources and computing grids. Storage 
resources include storage elements with SRM interface and most of the popular data access protocols (gridftp,
(s)ftp, http,...) are integrated as well. DIRAC does not provide a complex storage element, it includes however a 
Storage Element service which provides access to disk storage managed by a POSIX compliant file system.

\item[Services] \hfill \\
The DIRAC system is built around a set of loosely coupled services which keep the system state and
help to carry out workload and data management tasks. The services are passive components which
are only reacting to the requests of their clients possibly soliciting other services in order to
accomplish the requests. Each service has typically a MySQL database backend to store the state
information. The services accept incoming connections from various clients. These can be user interfaces,
agents or running jobs. 

\item[Agents] \hfill \\
Agents are light and easy to deploy software components built around a unified framework. They are active
components, usually deployed close to the corresponding services, watching for changes in the service states and 
reacting accordingly by initiating actions like job submission or result retrieval. 

\item[Interfaces] \hfill \\
The DIRAC functionality is exposed to the system developers and to the users in a variety of ways. 
	\begin{itemize}
	\item The DIRAC programming language is python, so the programming interfaces (APIs) are provided in this 		
		language
	\item For users the DIRAC functionality is available through a command line interface. Some DIRAC subsystems 		
		have specialized shells to work with.
	\item DIRAC also provides a web interface suitable for monitoring and managing the system behavior.
	\end{itemize}

\end{description}

\subsection{Framework}
DIRAC’s logic is built on top of basic services that provide transversal
functionality to DIRAC Systems \cite{DISET}. The set of basic services that form the core framework for
DIRAC are: 

\begin{description}

\item[DISET (DIRAC Secure Transport) secure communication protocol] 
	is the DIRAC secure transport layer. DISET was created by enhancing the XML-RPC
	protocol with a secure layer with GSI compliant authentication mechanism \cite{DISET2}.
	It takes care of all the communication needs between DIRAC components. 
	Communication is done using plain TCP/IP when secure connections are not needed.
	
\item[Configuration System] 
	is built to provide static configuration parameters to all the distributed DIRAC components. Being able to
	distribute the configuration is critical to DIRAC. If the configuration data is not available, 
	DIRAC systems can not work. For maximum reliability it is organized as a single master service where 
	all the parameter updates are done and multiple read-only slave services which are distributed geographically.

\item[Web Portal]  
	is the standard way to present information to visitors. It provides authentication based on user grid
	credentials and user groups: ordinary users can see their jobs and have minimal interaction with
	them, administrators can change the global configuration in a visual way. All the monitoring and control
	tools of a DIRAC system are exported through the Web portal which makes them uniform for user
	working in different environment and on different platforms.
	
\item[Logging System] 
	is a uniform logging facility for all the DIRAC components.

\item[Monitoring System]
	collects activity reports from all the DIRAC services and some agents. Together
	with the Logging Service, it provides a complete view of the health of the system for the managers.

\end{description}

Other widely used systems include the Workload Management System, able to manage simultaneously computing
tasks for a given user community \cite{WMS}, Request Management System managing simple operations that 
are performed asynchronously on behalf of users, and others. 

\section{DIRAC Data Management System}

The DIRAC middleware solution covers the whole range of tools to handle data management tasks. The low level 
tools including many clients for various types of storage. DIRAC provides only a simple Storage Element service 
accessible through the DISET protocol with a disk file system back-end. Otherwise, clients of all 
the mostly used storage systems are provided, exposing to the users a uniform API. Many experiments use the LCG
File Catalog (LFC) through a client API which connects it to the DIRAC system, others are using 
the DIRACs own File Catalog, discussed later. The Replica Manager class which encapsulates all the basic file 
management operations: uploading, replication, registration. The Replica Manager masks the diversities 
of different storage systems and can handle several file catalogs at the same time. 
The higher level Data Management components include an automatic Data Distribution System and
a Data Integrity Checking System. The Data Distribution System allows defining destination storages
for all the types of data used by a VO before the data actually become available. Once the first replicas
of the data to be distributed are registered in the File Catalog, the replication requests are formed and
sent for execution automatically. The Data Integrity Checking System is checking the integrity of data stored and
registered in various storage systems and file catalogs. 

\section{DIRAC File Catalog}

Any middleware solution covering the problem of Data Management needs a Replica Catalog, to keep track of 
file replication, and a File Catalog, to help the user find the files he wishes to use. The DIRAC 
couples these two services into one introducing the DIRAC File Catalog (DFC) \cite{DFC}. Implemented using the DISET 
framework, the access to DFC is done with an efficient secure protocol compliant with the grid security standards.
The protocol allows also defining access rules for various DFC functionalities based on the user roles. The
clients are getting the service access details from the common Configuration Service. The DFC behaviour
is monitored using the standard DIRAC Monitoring System and the errors in the service functioning are
reported to the Logging System. A MySQL database is used to store all the DFC contents. %TODO figure MYSQL schema
The file data in DFC is stored in a hierarchical directory structure. All the standard file metadata 
commonly found in file systems are provided (file size, owner, creation date...). The access control 
is implemented as a number of plug-ins that can be chosen according to the needs of a given user community.


\subsection{Replica Catalog}

The replica information can be stored in 2 different ways. In the first case, the full Physical File
Names (PFN) are stored as complete access URLs. In the second case, the DFC exploits a convention
that the PFNs are containing the corresponding Logical File Name (LFN) as its trailing part. This is a very common
convention allowing easily establish the correspondence between the physical files and entries in the catalogs, 
which helps when checking the data integrity. There is also no need to store the full PFN in the catalog, when
the information about the wanted storage element is available, the PFN can be constructed on the fly and when 
the SE description changes, for example, its access point, it is enough to change the SE description in the DIRAC 
Configuration System to continue receiving correct PFNs. Support for managing various user and group usage 
quotas is also implemented. Another feature requested based on the experience in 
the HEP experiments is to store and provide information about ancestor-descendent relations between files. 
DFC supports this basic data provenance information. 

\subsection{Metadata Catalog}

The metadata catalog keeps track of nonstandard metadata. It allows defining arbitrary user metadata 
associated with its files and directories as arbitrary key-value pairs. The catalog administrators can 
declare certain keys, or tags, as indexes. In the current implementation, whenever an index is being declaired, 
another table in the MySQL database is created to store the values of the respective metadata. The indexed 
metadata can be used in the file lookup operations. The metadata values are inherited in the directory structure.
The value types include numbers, strings or time stamps. Storing the metadata in the same directory hierarchy 
as the file logical name space allows keeping the same schema for both replica and metadata information.