\chapter{NoSQL Integration into DFC}
\label{chap:NoSQL}

To prove the concept of connecting the metadata part of the DIRAC File Catalog to a NoSQL database the 
modules \texttt{fmeta} and \texttt{dmeta} that manage the metadata in the FileCatalog service were modified.
As discussed in the previous chapter, the best available NoSQL database for this particular purpose is 
Elasticsearch. 

\section{Query Strategy}
The current query search algorithm cuts the query per element of the disjunction. With each element the Catalog
finds directories satisfying the directory part of the query, then it finds the files satisfying the file part. 
With the list of files and list of directories it filters out the files that are not in the sub-tree of one of the
satisfying directories (the algorithm is described by algorithm \ref{alg:find}).

\begin{algorithm}[h]
\caption{Find files satisfying query}
\label{alg:find}
\begin{algorithmic}[1]
\Function{Find}{queryList}
	\State \textit{outFiles} = [  ]
	\ForAll{\textit{element} in \textit{queryList}}
		\State dirMeta $\leftarrow$ getDirectoryMeta(\textit{element})
		\State \textit{dirs} $\leftarrow$ findDirectories(\textit{dirMeta}) \Comment{including sub-trees}
		\State fileMeta $\leftarrow$ getFileMeta(\textit{element})
		\State \textit{files} $\leftarrow$ findFiles(\textit{fileMeta})
		\State \textit{files} $\leftarrow$ dropFilesNotInDirs(\textit{files}, \textit{dirs})
		\State \textit{outFiles}.append(\textit{files})
	\EndFor
	\State \textbf{return} removeDuplicates(outFiles)
\EndFunction
\end{algorithmic}
\end{algorithm}

The procedure is so complicated, because the metadata for directories and for files are managed by different 
modules (in the algorithm, the function \texttt{findDirectories} is in module \texttt{dmeta} and \texttt{findFiles} 
is handled by \texttt{fmeta}). This can result in unnecessary fetching of files, which are then dropped because
they happen to be located in a directory that does not satisfy the query. This is especially relevant because as
the tests done by this project suggest the query speed of Elasticsearch unlike MySQL depends mainly on the
number of hits it has to retrieve and
not on the complexity of the query (see figure \ref{fig:wide}). 

The solution could be letting database engine deal 
with the complexity of the search algorithm. This would mean storing all the metadata associated with a file (its' 
own as well as the ones inherited from the directory structure) in the files document. This would increase
disk space (but as table \ref{tab:allDbSizes} suggests, Elasticsearch is much more space efficient when compared 
to MySQL) as well as the difficulty of metadata management. The inheritance of metadata through the directory 
structure has to be maintained when setting and unsetting them. The result would be that finding files would 
only consist of translating the metaquery from the internal representation to an Elasticsearch compatible form 
(which is now done per element in functions \texttt{findDirectories} and \texttt{findFiles}) and submitting the
query. What would be sacrificed is the fact that directory and file metadata can be handled in a completely 
different fashion.

\section{The New Implementation}

A new module has been developed to create a specialized interface between the database and the metadata catalog. This 
also minimized the changes that had to be done on the current metadata managing modules (the practice where one
module manages directory metadata and another manages file metadata was preserved). As discussed above, the 
module stores all the metadata in one document to improve the speed of queries. Setting and un-setting directory 
metadata is a rather complicate procedure, which requires fetching all the documents associated with files in
the particular sub-tree and updating them with the new metadata (respectively removing the one removed from the 
directory). 

Removing a property from a document in Elasticsearch is not one of the basically supported operations. There are 
two ways to implement it: 
\begin{enumerate}
\item Get the whole document from the database, remove the property and insert it again.
\item Submit a script that deletes the property from the document inside the database in a update-like operation.
\end{enumerate}
Because the scripting has to be enabled by the database administrator and it is not guaranteed for the DIRAC 
administrator to have administrator access to the database as well, the first approach was chosen. This means that
removing a metadata from a directory that has lots of files in its sub-tree is a non-trivial operation and shall
be executed as rarely as possible. Updating and setting new metadata can be done by a simple command.

For searching the metadata managing modules from the file catalog had to be changed. The new algorithm simply 
converts the query to the correct format, submits it and reads the result.