\chapter{User Documentation}
\label{chap:user}

To start working with DIRAC 
(or the Grid in general), the user should join some 
grid Virtual Organization and obtain a Grid Certificate. 
Before a user can work with DIRAC, the user’s certificate proxy 
should be initialized and uploaded to the DIRAC ProxyManager Service \cite{DISET}.
This is done by running command \texttt{dirac-proxy-init}. After that,
all the DIRACs' functionality is ready for use and all the actions will
be signed by the users certificate.

The primary interface to the DIRACs' file and metadata catalog
is the command-line interface provided by script
\texttt{dirac-dms-filecatalog-cli}. Commands that use the modules 
implemented by this project are \texttt{meta}, \texttt{find} 
and \texttt{dataset}.

\section{Command Meta}
Command used for all the metadata manipulating operations.
The sub-commands of command \texttt{meta} are: \texttt{index}, \texttt{get},
\texttt{set}, \texttt{remove} and \texttt{show}.

\subsection{index}
Usage: \texttt{meta index [-d|-f|-r] <metaname> <metatype>}

This command adds new a metadata name to available metadata names.
The user then can assign a value to the metaname for a concrete file
or directory. The parameters \texttt{-f} and \texttt{-d} specify,
whether the new metaname will be associated with files or directories,
respectively. The \texttt{-r} parameter is used, when the user
wants to remove the metadata name. \texttt{<metatype>} can be
one of the following: int, float, string, date. The user has to be
in the \texttt{dirac\_admin} group to be allowed to use this
command.

\subsection{set}
Usage: \texttt{meta set <path> <metaname> <metavalue>}

This command  sets a value to a metaname for a directory or a file specified by
the path parameter. 

\subsection{get}

Usage: \texttt{meta get [<path>]}

This command gets all user defined metadata for a directory or file specified by
the path parameter. When getting metadata for a file, it also gets
all the inherited directory metadata. When getting metadata for a
directory, the output specifies which are defined on that concrete
directory and which are inherited:
\begin{lstlisting}[language=json]
>meta get dir1
         !testDirStr : foo
          *unitFloat : 1.5
         *testDirInt : 1
\end{lstlisting}
The metadata prefixed with the \textbf{!} mark are defined on the
directory, others (prefixed with \textbf{*}) are inherited.

\subsection{remove}

Usage: \texttt{meta [remove|rm] <path> <metaname>}

This command removes the previously defined metaname for a directory or a file
specified by the path parameter.

\subsection{show}

Usage: \texttt{meta show}

This command shows all the available metadata names, divides them between the
file and the directory metadata and supplies the type.

\section{Command Find}

Usage: \texttt{find [-q] [-D] <path> <metaQuery>}

This command finds all files satisfying the given metadata query. When the
command is invoked it parses the query and, unless the \texttt{-q}
parameter is used, prints it out in its internal
representation. After the query results return, the list of all
the files satisfying it is printed out. When the \texttt{-D}
parameter is used, the command prints only the directories
that contain the files satisfying the metaquery.

\section{Command Dataset}
Command used for all the operations involving the dataset
functionality. Its sub-commands are \texttt{add, annotate, check,
download, files, freeze, locate, overlap, release, replicate, rm,
show, status} and \texttt{update}.

\subsection{add}

Usage: \texttt{dataset add [-f] <dataset\_name> <meta\_query>}

This command adds a new dataset containing all files satisfying the query
supplied by the \texttt{<meta\_query>} parameter. In the
\texttt{<dataset\_name>} the user specifies the path where the
dataset should be located. When parameter \texttt{-f} in used, the
dataset is immediately frozen.

\subsection{annotate}

Usage: \texttt{dataset annotate [-r] <dataset\_name> <annotation>}

This command adds annotation to a dataset. There is no indexing involving
annotations, they are just for the user. The annotation is a string
up to 512 characters long. When the \texttt{-r} parameters is used,
the annotation is removed from the specified dataset. A dataset can
have only one annotation.

\subsection{check}

Usage: \texttt{dataset check <dataset\_name> [<dataset\_name>]*}

This command checks the correctness of the cached information about the dataset.
It can be supplied with one dataset, or a list of dataset names
separated with spaces. The information can be outdated by recent
file addition or removal in the file catalog and can be updated
using the sub-command \texttt{update}.

\subsection{download}

Usage: \texttt{dataset} \texttt{download} \texttt{<dataset\_name>}
\texttt{[-d <target\_dir>]} \texttt{[<percentage>]}

This command invokes download of the dataset files. If the \texttt{<percentage>}
parameter is used, a subset of the datasets files are downloaded.
The size of the downloaded part is approximately the inserted percentage of the size
of the whole dataset.
Unless the \texttt{-d} parameter supplies the
target directory, the current working directory is used. The
command creates a directory with the same name as the dataset and
all the files are saved there.

\subsection{files}

Usage: \texttt{dataset files <dataset\_name>}

This command lists all the files that the specified dataset groups. In case
of a frozen dataset, the files saved in the database associated
to the dataset are listed. When the dataset is dynamic, the command
returns the same result, as using the \texttt{find} command with
the datasets metaquery.

\subsection{freeze}

Usage: \texttt{dataset freeze <dataset\_name>}

This command changes the status of a dataset to frozen. All the files that
satisfy the datasets metaquery at the moment of the command
invocation are associated with the dataset. For releasing the
dataset, the sub-command \texttt{release} is used.

\subsection{locate}

Usage: \texttt{dataset locate <dataset\_name>}

This command shows the distribution of the dataset files over the storage
elements providing the absolute size and percentage of the dataset
size used per storage element.

\subsection{overlap}

Usage: \texttt{dataset overlap <dataset\_name1> <dataset\_name2> }

This command checks if the two datasets have the same files. The metaqueries are
checked first so that when there cannot be files satisfying both,
the check does not compare lists of files.

\pagebreak

\subsection{release}

Usage: \texttt{dataset release <dataset\_name>}

This command changes the status of a dataset to dynamic. All the records
associating concrete files to the dataset are deleted from the
database. For freezing the dataset, the sub-command \texttt{freeze}
is used. However, if there were files added to the file catalog that
satisfy the datasets metaquery after the dataset was frozen, the
subset of files satisfying the metaquery not including those files
cannot be recreated after releasing the dataset.

\subsection{replicate}

Usage: \texttt{dataset replicate <dataset\_name> <SE>}

This command initiates a bulk replication of the datasets files to a storage
element specified by the parameter \texttt{<SE>}. The replication
is handled by the Request Management System.

\subsection{rm}

Usage: \texttt{dataset rm <dataset\_name>}

This command deletes the dataset from the file catalog. If the dataset is
frozen, its files are cross-checked with all other frozen datasets
and if there are some files that are grouped by the deleted dataset
only, the user is offered the option to delete those files from the
File Catalog. The deletion is handled by the Request Management
System.

\subsection{show}

Usage: \texttt{dataset show [-l] [<dataset\_name>]}

This command lists the names of all the existing datasets. When the \texttt{-l}
option is used, other data about the datasets are printed as well.
When a \texttt{<dataset\_name>} is provided, the command restricts
itself to this dataset.

\subsection{update}

Usage: \texttt{dataset update <dataset\_name> [<dataset\_name>]*}

This command updates the cached parameters for a dataset or for a space-separated 
list of datasets.

\pagebreak

\subsection{status}

Usage: \texttt{dataset status <dataset\_name> [<dataset\_name>]*}

This command prints details about a specified dataset or a space-separated list of
datasets.

\begin{lstlisting}[language=json]
> dataset status testDataset
testDataset:
========
    Key              Value
======================================================
  1 NumberOfFiles    13
  2 MetaQuery        testDirInt = 1
  3 Status           Dynamic
  4 DatasetHash      A89F08D23140960BDC5021532EF8CC0A
  5 TotalSize        61562
  6 UID              2
  7 DirID            5
  8 OwnerGroup       dirac_admin
  9 Owner            madam
 10 GID              1
 11 Mode             509
 12 ModificationDate 2015-09-24 11:50:28
 13 CreationDate     2015-09-24 11:50:28
 14 DatasetID        71
\end{lstlisting}

