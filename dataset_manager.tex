\chapter{The DatasetManager}
\label{chap:Dataset}

As mentioned before, the DIRAC software architecture is based on a set of distributed,
collaborating services, following the paradigm of a Service
Oriented Architecture (SOA). The services themselves consist of
modules. DatasetManager is one of the modules used by the File Catalog, where the other include
e.g. FileMetadata module, DirectoryTree module etc. 

The dataset is a set of files grouped together based on their common metadata. To accomplish
that, each dataset is associated with a metaquery and all the files that it contains are
satisfying it. Moreover there are two types of datasets: a dynamic dataset and 
a frozen (static) one. A dynamic dataset is exactly what is described above: a set of all the
files that satisfy a given metaquery. The dynamic dataset changes every time a file satisfying
its metaquery is added to the file catalog. A frozen dataset is a snapshot of a dynamic dataset. The
user can freeze a dataset -- the files of which he is using -- in order to maintain the set of files for
further evaluation or cross-checking the results of his work with other users over the same
set of files. When a frozen dataset is later deleted, the user is asked whether the files, that 
are exclusively contained in this dataset should be deleted. 

\section{Previous Status}
Before the start of this project, a prototype of the DatasetManager module was already 
in place. It complied with the demands requiring, what the structure of the database 
should look like and what is the 
expected behavior of the datasets, but many crucial features (like e.g. deleting files when a 
frozen dataset is deleted, dataset replication and overlapping and other) and properties (identifying
the dataset based on its' location in the directory tree) were missing.

\section{Implementation Details}

The DatasetManager is  one of the modules in the file catalog. The class offers
multiple methods handling the datasets and exposing them to other modules. The state of the 
datasets is saved in a database, the class itself holds only an instance of the database interface
object, so restarting the FileCatalog service is done without any problems. The module handles mainly
basic CRUD operations with some additions.

\subsection{Releasing and Freezing a Dataset}

When a dataset is created, an entry is inserted to the database. It contains the dataset name, id, metaquery, 
status, id of the directory it is located in, UID and GID of the user who created it and some other 
properties. When a dataset is frozen, the query is evaluated and all the files, which the
dataset contains, are associated with it (the records are inserted in the table \texttt{FC\_MetaDatasetFiles}).
Releasing the dataset only deletes these records and updates the cached properties. 

When a dataset is deleted, its record in the database is removed. Moreover when the deleted dataset is frozen
it is requested it would be deleted with the files it contains not including files contained by another frozen
dataset. This is done using one select

\begin{minted}{sql}
SELECT FileID 
FROM FC_MetaDatasetFiles 
WHERE FileID IN (
	SELECT FileID 
	FROM FC_MetaDatasetFiles 
	WHERE DatasetID=X
) 
GROUP BY FileID 
HAVING Count(*)=1;
\end{minted}

\noindent where in place of the \texttt{X} symbol the id of the deleted dataset is inserted. The user is then 
asked, whether he really wants  delete those files from the file catalog. When he agrees, a request to
the DIRACs Request Management System (RMS) is created. This system ensures that all the files
are deleted from all the storage elements as well as from the file catalog database, 
which prevents memory leaks. The RMS is also used when dataset replication is executed.

\subsection{Dataset Overlapping}

Another important feature not included in the previous implementation was the determination of  
dataset overlapping. In the first step the metaqueries of the two datasets are combined
into a conjunction and the check proceeds only if the conjunction is a valid metaquery.
The procedure of the second step depends on the statuses of the datasets. When the two
datasets in question are dynamic, the files satisfying the conjunction of their metaqueries
are their overlap. When comparing a dynamic and a frozen dataset, the files satisfying 
the conjunction are compared to the list of the frozen datasets files. Finally, when 
two frozen datasets are checked, the sets of their files are simply compared and if 
the intersection is empty they do not overlap.